\chapter*{关于本书}\label{about}
\addcontentsline{toc}{chapter}{关于本书}

嗨,欢迎光临!Elixir 是一种构建在 Erlang虚拟机上的函数式编程语言。
它结合了 Ruby 的生产力和表现力以及 Erlang的并发性和容错性。
Elixir 充分利用了 Erlang 强大的 OTP库,许多开发者认为这是 Erlang伟大的源泉,因此你可以立即拥有成熟、专业品质的功能。
Elixir对函数式编程的支持使其成为高度分布式的事件驱动应用程序(如物联网系统)的绝佳选择。

本书尊重您的时间,旨在尽快带您入门 Elixir 和OTP。然而,它期望您投入必要的工作量来掌握所有各种概念。因此,只当您尝试实例并进行实验时,本书才能发挥最佳的效果。如果您遇到困难,请不要烦恼 - Elixir 社区非常欢迎新手!

\section*{书籍组织方式}

本书分为三部分,共十一章,还有一个附录。第一部分涵盖了 Elixir 和 OTP的基础知识。

第一章介绍 Elixir,它与其父语言 Erlang的不同之处,与其他语言的比较,以及 Elixir 和 OTP 的用例。

第二章带您快速了解 Elixir。您将编写您的第一个 Elixir程序,并熟悉语言基础。

第三章介绍进程,Elixir 的并发单位。您将了解 Actor并发模型,以及如何使用进程发送和接收消息。然后,您将组装一个示例程序,以查看并发进程的实际操作。

第四章介绍 OTP,这是 Elixir 从 Erlang 继承的杀手级特性之一。您将了解 OTP的哲学,以及作为 Elixir 程序员将使用的 OTP 中最重要的部分。您将了解 OTP行为的工作方式,并使用 GenServer 行为构建您的第一个 Elixir/OTP 应用程序- 一个与第三方服务通信的天气程序。

第二部分涵盖了 Elixir 和 OTP的容错和分布式方面。第五章查看处理错误的基本原理,特别是在并发环境中。您将了解Erlang VM在处理进程崩溃时所采取的独特方法。您还将体验构建自己的监督进程(类似于 Supervisor OTP 行为),然后使用真实的 Supervisor。

第六章全面讨论 Supervisor OTP 行为和容错。您将了解 Erlang的``让它崩溃''哲学。本章介绍了使用前几章建立的技能的工作池应用程序。

第七章继续讨论工作池,我们在其中添加更多功能,使其成为更全面和现实的工作池应用程序。在此过程中,您将学习如何构建非平凡的监督层次结构,并学习如何动态创建监督和工作进程。

第八章考察分布式及其在负载平衡方面的帮助。它指导您构建一个分布式负载平衡器。同时,您将学习如何在Elixir 中构建命令行程序。

第九章继续讨论分布式,但这次我们着眼于故障转移和接管。这对于必须对故障具有弹性的任何非琐碎应用程序来说都是绝对关键的。您将构建一个既具有容错性又具有分布式特性的Chuck Norris 笑话服务。

第三部分涵盖了 Elixir中的类型规范、基于属性的测试和并发测试。我们将研究三种工具,即Dialyzer、QuickCheck 和Concuerror,并查看这些工具如何帮助我们编写更好、更可靠的 Elixir 代码。

附录提供了在您的机器上设置 Erlang 和 Elixir 的说明。

\section*{谁应该阅读这本书}

您手头没有太多时间。您想看看 Elixir 是什么,并尽快动手实践。

我假设您熟悉终端操作并具有一些编程经验。

虽然事先了解 Elixir 和 Erlang当然有帮助,但绝不是必须的。然而,本书并不打算作为 Elixir的参考书。您应该知道如何自行查找文档。

您也不排斥变化。Elixir变化很快。但是,您正在阅读这本书,所以我假设这对您来说不是问题。

\section*{如何阅读这本书}

从前到后。本书的进展是线性的,虽然早期章节或多或少是独立的,但后面的章节是基于前面的章节构建的。
一些章节可能需要重读,所以不要认为您应该在第一次阅读时就理解所有概念。

我最喜欢的编程书是那些鼓励您尝试代码的书。概念总是以这种方式更好地沉淀。
在这本书中,我也力求达到同样的效果。没有什么比动手实践更有益了。
一些章节的结尾有练习。\textbf{做它们!}在头脑清晰、终端打开、渴望学习有趣且有价值的东西的情况下,这本书最有用。

\section*{获取示例代码}

最新的书籍代码托管在GitHub仓库: \url{https://github.com/benjamintanweihao/the-little-elixir-otp-guidebook-code}。

\section*{关于作者}

Benjamin Tan Wei Hao 是新加坡 Pivotal Labs的软件工程师。由于害怕变得无关紧要,他总是试图赶上他不断增长的阅读清单。他喜欢参加Ruby 会议并谈论 Elixir。

他是 LeanPub 上一本自行出版的书《Ruby 闭包书》的作者。他还为 SitePoint的 Ruby 栏目撰稿,并试图偶尔插入一篇 Elixir文章。在他充裕的空闲时间里,他在 \url{benjamintan.io} 上写博客。
