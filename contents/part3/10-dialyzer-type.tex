\chapter{Dialyzer和类型规范}\label{chapt:dialyzer}

本章内容包括:

\begin{itemize}

\item  什么是Dialyzer以及它是如何工作的
\item  使用Dialyzer发现代码中的不一致
\item  编写类型规范和定义自己的类型
\end{itemize}

根据你的倾向,类型的提及可能会让你欣喜若狂或者退避三舍。作为一种动态类型语言,Elixir免除了你在代码库中大量使用类型的需求,这一点类似于Haskell。有人可能会认为这导致了更快的开发周期。然而,Elixir程序员不应该过于自满。静态类型语言可以在\emph{编译}时捕捉到一整类错误,而动态语言只能在\emph{运行时}捕捉到这些错误。

幸运的是,语言中内置的容错功能试图拯救我们自己。没有这些功能的语言(Ruby,我在看你)将会直接崩溃。然而,我们有责任尽可能地使我们的软件可靠。在本章中,我们将学习如何利用类型来实现这一点。

我们将了解Dialyzer,这是一个与Erlang发行版捆绑在一起的工具。这个强大的工具用于消除某些类别的软件错误。最棒的部分是,你不需要对你的代码做任何特别的处理。

你将了解一些关于Dialyzer背后有趣的理论。这将帮助你解读它的(有时是隐晦的)错误信息。你还将理解为什么Dialyzer不是解决所有类型问题的灵丹妙药。

在本章的最后部分,我们将学习如何通过在代码中添加类型,使Dialyzer更好地寻找错误。在本章结束时,你将学会如何将Dialyzer作为开发工作流的一部分。

Dialyzer的命名者因为这个与电信相关的缩写而值得加薪。Dialyzer代表了Erlang的不一致性分析器。Dialyzer是一个帮助你发现代码中不一致之处的工具。具体是什么类型的不一致呢?以下是一个列表:

\begin{itemize}

\item  类型错误
\item  引发异常的代码
\item  无法满足的条件
\item  冗余代码
\item  竞态条件
\end{itemize}

我们将很快亲自看到Dialyzer是如何发现这些不一致的。在此之前,了解Dialyzer的内部工作原理是有帮助的。

\section{10.1 Dialyzer是如何工作的}

静态语言可以在编译时捕捉潜在的错误。动态语言的本质意味着它们只能在运行时检测到这些错误。Dialyzer试图将静态类型检查器的一些优点带给像Elixir/Erlang这样的动态语言。

Dialyzer的主要目标之一是不干扰现有程序。这意味着不应该期望任何Erlang(和Elixir)程序员重写代码来适应Dialyzer。

这导致了一个非常好的结果:你不需要提供给Dialyzer任何额外的信息,它就能完成它的工作。这并不是说你\emph{不能}这样做。事实上,正如你稍后将看到的,你可以提供额外的类型信息,让Dialyzer在寻找不一致时做得更好。

\section{10.2 成功类型}

Dialyzer使用\emph{成功类型}的概念来收集和推断类型信息。了解Dialyzer

如何工作是值得的。要理解成功类型是什么,我们需要了解一点关于Elixir类型系统的知识。

像Elixir这样的动态语言需要一个比静态类型系统更宽松的类型系统,因为函数可能会接受多种类型的参数。

例如,让我们看看布尔``和''函数。在像Haskell这样的静态语言中,\texttt{and}函数可以这样实现:

\begin{code}{Haskell中的布尔与运算}
  \begin{minted}[linenos]{Haskell}
    and :: Bool -> Bool -> Bool
    and x y | x == True && y == True = True
                | otherwise = False
  \end{minted}
  \label{lst:boolean-in-haskell}
\end{code}

第一行 \texttt{and :: Bool -> Bool -> Bool}是函数的类型参数。
它表明 \texttt{and}是一个接受两个布尔值作为参数并返回一个布尔值的函数。
如果类型检查器看到任何非布尔值作为输入到\texttt{and},你的程序将无法通过编译。Elixir版本会是什么样子呢?

\begin{code}{在Elixir中实现的布尔与运算}
\begin{minted}[linenos]{elixir} 
defmodule MyBoolean do

  def and(true, true) do
    true
  end
  
  def and(false, _) do
    false
  end
  
  def and(_, false) do
    false
  end
  
end
\end{minted}
\label{lst:boolean-in-elixir}
\end{code}

多亏了模式匹配,我们可以将\texttt{and/2} 表示为三个函数子句。
什么是对\texttt{and/2}的有效参数?第一和第二个参数接受\texttt{true}, \texttt{false}和\texttt{\_}, 而返回值都是布尔值。

正如你已经知道的,``\_''意味着``任何东西''。
因此,以下是对\texttt{and/2}的完全合理的调用:

\begin{code}{在Elixir中实现的布尔与运算}
  \begin{minted}[linenos]{elixir} 
    MyBoolean.and(true, true) MyBoolean.and(false, "great success!")
    MyBoolean.and([1, 2, 3], false)
  \end{minted}
\end{code}

Haskell类型检查器不会允许像前面展示的Elixir程序,因为它不允许将``任何东西''作为一种类型。它无法处理这种不确定性。

另一方面,Dialyzer采用了一种不同的类型推断算法,称为成功类型。成功类型非常乐观。它总是假设你的所有函数都被正确使用。
因此,你的代码在被证明有罪之前是无辜的。

成功类型从\emph{过度估计}你的函数的有效输入和输出开始。
所以它从假设你的函数可以接受任何东西并返回任何东西开始。
然而,随着它更好地理解你的代码,它会生成\emph{约束}。
这些约束反过来将限制输入值以及对应的输出。

例如,如果它看到 \texttt{x + y},那么\texttt{x} 和 \texttt{y}肯定是数字。
像 \texttt{is\_atom(z)}这样的守卫也提供了额外的约束。
一旦生成了约束,就是解决它们的时候了,就像解谜一样。
谜题的解答就是函数的成功类型。
相反,如果没有找到解决方案,约束是\emph{不可满足的},你手头就有一个类型违规。

然而,重要的是要意识到,因为Dialyzer总是假设你的代码是正确的,所以它\emph{不}保证你的代码是类型安全的。现在,在你起身离开房间之前,由此产生了一个非常好的属性。如果Dialyzer发现了什么问题,那么它\emph{肯定}是对的。所以Dialyzer的第一课是这样的:

\begin{note}{Dialyzer永远不会出错!}
Dialyzer 在它说你的代码有问题时,始终是正确的。
\end{note}


这就是为什么当Dialyzer表示你的代码有问题时,它是100\%正确的。
更严格的类型检查器从假设你的代码是错误的开始,你的代码必须成功通过类型检查才能允许编译。这也意味着你的代码(或多或少)被保证是类型安全的。

所以再次强调:Dialyzer\emph{不会}(或永远不会)发现所有类型违规。
然而,如果它发现了问题,那么你的代码\emph{肯定}存在问题。
现在我们对成功类型(success typings)的工作方式有了一些背景知识,让我们转向了解Elixir中的类型。

\section{揭示Elixir中的类型,第一部分}

我们一直在使用Elixir,但并没有太多强调确切的类型。在本节和下一节中,我们将稍微更加关注这一点。

从Elixir1.2开始,有一个非常方便的工具在\texttt{iex}中可以打印给定数据类型的信息,称为\texttt{i/1}。
例如,\texttt{"ohai"}和\texttt{'ohai'}之间有什么区别(注意分别使用双引号和单引号)?让我们来找出答案:

\begin{code}{使用i/1揭示Elixir字符串的类型}
\begin{minted}[linenos]{elixir}
iex(1)> i("ohai")
  Term
    "ohai"
  Data type
    BitString
  Byte size
    4
  Description
    This is a string: a UTF-8 encoded binary. It's printed surrounded by
    "double quotes" because all UTF-8 encoded code points in it are printable.
  Raw representation
    <<111, 104, 97, 105>>
  Reference modules
    String, :binary
  Implemented protocols
    Collectable, IEx.Info, Inspect, List.Chars, String.Chars
\end{minted}
% \label{lst:id}
\end{code}

现在让我们对比一下\texttt{'ohai'}:

\begin{code}{使用i/1揭示字符列表的类型}

\begin{minted}[linenos]{elixir}
iex(2)> i('ohei')
  Term
    ~c"ohei"
  Data type
    List
  Description
    This is a list of integers that is printed using the `~c` sigil syntax,
    defined by the `Kernel.sigil_c/2` macro, because all the integers in it
    represent printable ASCII characters. Conventionally, a list of Unicode
    code points is known as a charlist and a list of ASCII characters is a
    subset of it.
  Raw representation
    [111, 104, 101, 105]
  Reference modules
    List
  Implemented protocols
    Collectable, Enumerable, IEx.Info, Inspect, List.Chars, String.Chars
\end{minted}
% \label{lst:id}
\end{code}

下次如果你遇到类型错误并感到困惑,立即使用\texttt{i/1}工具。

\section{开始使用Dialyzer}

Dialyzer可以使用Erlang源代码或调试编译的BEAM字节码。显然,这让我们只能选择后者。这意味着在我们运行Dialyzer之前,必须记得先执行\texttt{mix compile}。

\begin{note}{记得先编译!} 
  
  自从开始使用Dialyzer以来,我已经忘记了多少次这个步骤。幸运的是,一旦我发现了Dialyxir(稍后你会看到),我就不再需要手动编译我的代码了。
  
\end{note}

Dialyzer随Erlang发行版安装,并且存在作为命令行程序:

\begin{code}{}\begin{minted}[linenos]{elixir}
  % dialyzer
      Checking whether the PLT /Users/benjamintan/.dialyzer_plt is up-to-date... 
      dialyzer: Could not find the PLT: /Users/benjamintan/.dialyzer_plt Use the options: 
      --build_plt   to build a new PLT; or 
      --add_to_plt  to add to an existing PLT
      
      For example, use a command like the following: 
         dialyzer --build_plt --apps erts kernel stdlib mnesia Note that building a PLT such as the above may take 20 mins or so 

      If you later need information about other applications, say crypto, you can extend the PLT by the command: 
        dialyzer --add_to_plt --apps crypto
      For applications that are not in Erlang/OTP use an absolute file name.
\end{minted}
% \label{lst:id}
\end{code}

太好了,我们已经确信Dialyzer确实已安装。但是这个\emph{PLT}是什么,Dialyzer正在尝试搜索什么呢?

\subsection{PLT:持久查找表}

PLT代表持久查找表(Persistent Lookup Table)。Dialyzer使用PLT来存储其分析结果。您还可以使用之前构建的PLT作为Dialyzer的起点。这变得很重要,因为任何非平凡的Elixir应用程序可能都会涉及OTP。如果我们对这样的应用程序运行Dialyzer,分析无疑会花费很长时间。

由于OTP库不会改变,我们总是可以构建一个``基础PLT'',只在我们的应用程序上运行Dialyzer,相比之下将会花费更短的时间。这的另一面是,一旦您升级了Erlang和/或Elixir,您必须记得重建PLT。

 \subsection{ Dialyxir(Dialyxir)}

传统上,运行 Dialyzer需要输入相当多的命令。幸好,由于程序员的懒惰,现在有一些库包含了\texttt{mix}任务,使我们的生活更加轻松。
我们将使用的一个库是\emph{Dialyxir}。
Dialyxir 包含了 \texttt{mix}任务,使得在 Elixir 项目中使用 Dialyzer 成为一种乐趣。

Dialyxir可以作为依赖安装(我们稍后会看到),也可以全局安装。我们首先全局安装Dialyxir,以便构建 PLT 表。
这不是绝对必要的,但当你不想将 Dialyxir安装为项目依赖时,这是很有用的:

\begin{code}{}\begin{minted}[linenos]{elixir}
git clone https://github.com/jeremyjh/dialyxir
cd dialyxir
mix archive.build
mix archive.install
\end{minted}
% \label{lst:id}
\end{code}

让我们开始使用 Dialyxir 吧!

\subsection{构建 PLT 表(建立 PLT表)}

如前所述,我们首先需要构建 PLT。令人高兴的是,Dialyxir 有一个 mix任务用于构建 PLT:

\begin{code}{}
\begin{minted}[linenos]{elixir}
% mix dialyzer.plt
\end{minted}
% \label{lst:id}
\end{code}

准备好咖啡,因为这需要一段时间:

\begin{code}{}
\begin{minted}[linenos]{elixir}
Starting PLT Core Build ... this will take awhile
dialyzer --output_plt /Users/benjamintan/.dialyxir_core_18_1.2.0-rc.1.plt --build_plt --apps erts kernel stdlib crypto public_key -r /usr/local/Cellar/elixir/HEAD/bin/../lib/elixir/../eex/ebin /usr/local/Cellar/elixir/HEAD/bin/../lib/elixir/../elixir/ebin /usr/local/Cellar/elixir/HEAD/bin/../lib/elixir/../ex_unit/ebin /usr/local/Cellar/elixir/HEAD/bin/../lib/elixir/../iex/ebin /usr/local/Cellar/elixir/HEAD/bin/../lib/elixir/../mix/ebin
...
cover:compile_beam_directory/1
cover:modules/0
cover:start/0
fprof:analyse/1
fprof:apply/3
fprof:profile/1
httpc:request/5
httpc:set_options/2
inets:start/2
inets:stop/2
leex:file/2
yecc:file/2
Unknown types:
compile:option/0
done in 2m33.16s
done (passed successfully)
\end{minted}
% \label{lst:id}
\end{code}

只要 PLT 构建成功,你就不必担心``未知类型''和其他警告。


\section{10.5 Dialyzer 可以检测的软件差异(Dialyzer能检测的软件问题)}

在本节中,我们将创建一个项目来进行实验。示例项目是一个简单的货币转换器,只能将新加坡元转换为美元。创建项目:

\begin{code}{}\begin{minted}[linenos]{elixir}
% mix new dialyzer_playground
\end{minted}
% \label{lst:id}
\end{code}

打开 \texttt{mix.exs} 并添加 Dialyxir:


\begin{code}{代码 10.4 添加 dialyxir 依赖(mix.exs)}

\begin{minted}[linenos]{elixir}
defmodule DialyzerPlayground.Mixfile do
  # ...

  defp deps do
    [{:dialyxir, "~> 0.3", only: [:dev]}]
  end
end
\end{minted}
% \label{lst:id}
\end{code}

像往常一样,记得运行\texttt{mix deps.get}。现在乐趣开始了!


\subsection{捕捉类型错误(捕获类型错误)}

我们从一个简单的示例开始,演示 Dialyzer 如何捕捉简单的类型错误。创建
\texttt{lib/bug\_1.ex}:

\begin{code}{代码 10.5 Cashy.Bug1有一个类型错误。你能发现吗?(lib/bug\_1.ex)}

\begin{minted}[linenos]{elixir}
defmodule Cashy.Bug1 do

  def convert(:sgd, :usd, amount) do
    {:ok, amount * 0.70}
  end

  def run do
    convert(:sg

d, :usd, :one_million_dollars)
  end
end
\end{minted}
% \label{lst:id}
\end{code}

\texttt{convert/3} 函数接受三个参数。前两个参数\emph{必须} 是原子 \texttt{:sgd} 和\texttt{:usd}。
\texttt{amount}被假定为一个数字,并用来计算从新加坡元到美元的汇率。相当直接的东西。

现在想象一下 \texttt{run/1}函数可能存在于另一个模块中。
不难想象有人错误地使用这个函数,比如将原子作为\texttt{convert/3} 的最后一个参数,而不是数字。

只有当 \texttt{run/1}被执行时,代码的问题才会显现。
否则,这个问题甚至可能不会浮现。值得注意的是,静态类型语言永远不会允许这样的代码。
对我们来说幸运的是,我们有Dialyzer!让我们运行 Dialyzer 看看会发生什么:

\begin{code}{}
\begin{minted}[linenos]{elixir}
% mix dialyzer
\end{minted}
% \label{lst:id}
\end{code}

这是输出:

\begin{code}{}
\begin{minted}[linenos]{elixir}
% mix dialyzer
Compiled lib/bug_1.ex
Generated dialyzer_playground app
...
Proceeding with analysis...
bug_1.ex:7: Function run/0 has no local return
bug_1.ex:8: The call 'Elixir.Cashy.Bug1':convert('sgd','usd','one_million_dollars') will never return since it differs in the 3rd argument from the success typing arguments: ('sgd','usd',number())
done in 0m1.00s
done (warnings were emitted)
\end{minted}
% \label{lst:id}
\end{code}

Dialyzer 发现了一个问题!Dialyzer
说的``无本地返回''意味着该函数肯定会失败。这通常意味着 Dialyzer
发现了一个类型错误,因此确定该函数永远不会返回。

正如它正确指出的,\texttt{convert/3}
因为我们给它的参数会导致 \texttt{ArithmeticError}
而永远不会返回。


\subsection{错误使用内置函数}

让我们检查另一种情况。创建文件 \texttt{lib/bug\_2.ex}:

\begin{code}{Cashy.Bug2 中错误使用了内置函数。(lib/bug\_2.ex)}

\begin{minted}[linenos]{elixir}
defmodule Cashy.Bug2 do
  def convert(:sgd, :usd, amount) do
    {:ok, amount * 0.70}
  end

  def convert(_, _, _) do
    {:error, :invalid_amount}
  end

  def run(amount) do
    case convert(:sgd, :usd, amount) do
      {:ok, amount} ->
        IO.puts("converted amount is #{amount}")

      {:error, reason} ->
        IO.puts("whoops, #{String.to_atom(reason)}")
    end
  end
end
\end{minted}
% \label{lst:id}
\end{code}

第一个函数子句与 \texttt{Cashy.Bug1}中的完全相同。
此外,还有一个捕获所有情况的子句,返回\mintinline{elixir}|{:error, :invalid_amount}|。
再次想象\texttt{run/1}被某处客户端代码调用。你能发现问题所在吗?让我们看看 Dialyzer 的说法:

\begin{code}{}
\begin{minted}[linenos]{elixir}
% mix dialyzer
...
bug_2.ex:18: 调用 erlang:binary_to_atom(reason@1::'invalid_amount','utf8') 违反了约定 (Binary,Encoding) -> atom() 当 is_subtype(Binary,binary()), is_subtype(Encoding,'latin1' | 'unicode' | 'utf8')
执行完毕耗时 0m1.02s(发出了警告)
\end{minted}
% \label{lst:id}
\end{code}

有趣!这里似乎有一个问题:

\texttt{erlang:binary\_to\_atom(reason@1::'invalid\_amount','utf8')}

似乎违反了某种形式的合约。在第18行,正如 Dialyzer 所指出的,我们调用了
\texttt{String.to\_atom/1}。看来这是问题的原因。\texttt{erlang:binary\_to\_atom/2}
正在寻找的合约是:

\texttt{(Binary,Encoding) -> atom()}

我们提供的输入是
\texttt{'invalid\_amount' 和 'utf8'},转换成
\texttt{(Atom, Encoding)}。仔细检查后,我们应该调用
\texttt{Atom.to\_string/1} 而不是
\texttt{String.to\_atom/1}。哎呀。

 \subsection{ 冗余代码}

冗余代码阻碍了可维护性。在某些情况下,Dialyzer可以分析代码路径并发现冗余代码。
\texttt{lib/bug\_3.ex}提供了这方面的一个例子:

\begin{code}{Cashy.Bug3 中有一个冗余的代码路径。(lib/bug\_3.ex)}

\begin{minted}[linenos]{elixir}
defmodule Cashy.Bug3 do
  def convert(:sgd, :usd, amount) when amount > 0 do
    {:ok, amount * 0.70}
  end

  def run(amount) do
    case convert(:sgd, :usd, amount) do
      amount when amount <= 0 ->
        IO.puts("whoops, should be more than zero")

      _ ->
        IO.puts("converted amount is #{amount}")
    end
  end
end
\end{minted}
% \label{lst:id}
\end{code}

这次,我们在 \texttt{convert/3}中添加了一个保护子句,确保只有在 \texttt{amount}大于零时才进行货币转换。
现在看看\texttt{run/1}。它有两个子句。其中一个处理\texttt{amount} 小于或等于零的情况。
第二个子句处理\texttt{amount} 更大的情况。Dialyzer 对此有何看法?

\begin{code}{}
\begin{minted}[linenos]{elixir}
% mix dialyzer
...
bug_3.ex:9: Guard test amount@2::{'ok',float()} =< 0 永远不会成功
执行完毕耗时 0m0.97s(发出了警告)
\end{minted}
% \label{lst:id}
\end{code}

Dialyzer 已经帮助我们识别了一些冗余代码!由于我们在\texttt{convert/3} 中有了保护子句,
我们可以确定\texttt{amount <= 0}的情况永远不会发生。
再次强调,这是一个简单的例子。
然而,不难想象程序员可能不了解这种行为,因此尝试覆盖所有情况,实际上这是冗余的。


\subsection{保护子句中的类型错误}

在使用保护子句的情况下可能会发生类型错误。保护子句限制了它们包裹的参数的类型。在下一个示例中,该参数是
\texttt{amount}。让我们看看\texttt{lib/bug\_4.ex}。你可能很容易发现问题所在:

\begin{code}{当 run/1 执行时会发生错误。你能猜出为什么吗?(lib/bug\_4.ex)}

\begin{minted}[linenos]{elixir}
defmodule Cashy.Bug4 do
  def convert(:sgd, :usd, amount) when is_float(amount) do
    {:ok, amount * 0.70}
  end

  def run do
    convert(:sgd, :usd, 10)
  end
end
\end{minted}
% \label{lst:id}
\end{code}

让 Dialyzer 发挥作用:

\begin{code}{}
\begin{minted}[linenos]{elixir}
% mix dialyzer
...
bug_4.ex:7: 函数 run/0 没有本地返回
bug_4.ex:8: 调用 'Elixir.Cashy.Bug4':convert('sgd','usd',10) 永远不会返回,因为它在第三个参数上与成功类型参数不符:('sgd','usd',float())
执行完毕耗时 0m0.97s(发出了警告)
\end{minted}
% \label{lst:id}
\end{code}

如果我们足够仔细,我们会意识到 \texttt{10} 不是\texttt{float()}类型,因此不符合保护子句。
关于保护子句的一个有趣之处在于,它们永远不会抛出异常,这正是它们的全部意义,因为你正在特别允许只有某些类型的输入。然而,这有时可能导致类似上面那种令人困惑的错误,当时看起来\texttt{10} 应该被允许通过保护子句。

\subsection{用一些间接方法让Dialyzer绊倒}

在本节的最后一个例子中,我们看一下\texttt{Cashy.Bug1}的一个略微修改的版本。创建\texttt{lib/bug\_5.ex}:

\begin{code}{Dialyzer将无法捕获此错误。 (lib/bug\_5.ex)}

\begin{minted}[linenos]{elixir}
defmodule Cashy.Bug5 do
  def convert(:sgd, :usd, amount) do
    amount * 0.70
  end

  def amount({:value, value}) do
    value
  end

  def run do
    convert(:sgd, :usd, amount({:value, :one_million_dollars}))
  end
end
\end{minted}
% \label{lst:id}
\end{code}

现在,看起来很明显,Dialyzer很可能会报告与\texttt{Cashy.Bug1}相同的错误。注意,我们在这里只是通过使\texttt{amount/1}成为一个函数调用,返回我们想要转换的金额的实际值,从而增加了一层间接性。让我们测试一下我们的假设:

\begin{code}{}
\begin{minted}[linenos]{elixir}
% mix dialyzer
...
Proceeding with analysis... done in 0m1.05s done (passed successfully)
\end{minted}
% \label{lst:id}
\end{code}

等等,什么?不幸的是,在这种情况下,由于这种间接性,Dialyzer无法检测到这种差异。这是一个完美的过渡到下一个关于类型规范的主题。我们将在那之后回到\texttt{Cashy.Bug5}。

\section{10.6 类型规范}

我们已经提到,Dialyzer可以在没有你的帮助下愉快地运行。我们已经向你展示了一些Dialyzer可以从\texttt{Cashy.Bug1}到\texttt{Cashy.Bug4}检测到的软件差异的例子。

然而,正如\texttt{Cashy.Bug5}所示,一切并非都是彩虹和独角兽。虽然Dialyzer可能会报告\texttt{passed successfully},但这并不意味着你的代码没有错误。有些情况下,Dialyzer无法完全自己检测到。

通过一些努力,我们可以帮助Dialyzer揭示难以检测的错误。我们通过添加\emph{类型规范},或者简称\emph{Typespecs}来做到这一点。

将类型规范添加到你的代码的另一个优点是,它可以作为一种文档形式。特别是对于动态语言,有时候并不明显什么是有效的输入,以及返回值的类型。在本节中,你将学习如何编写你自己的类型规范,不仅为了编写更好的文档,而且为了编写更可靠的代码。

\subsection{编写类型规范}

最好的方式是通过一些例子来向你展示如何使用类型规范。定义类型规范的格式是:

\texttt{@spec function\_name(type1, type2) :: return\_type}
这个格式应该是不言自明的。我们稍后会讲解什么是有效的类型值(\texttt{type1},\texttt{type2}和\texttt{return\_type})。下面是一些已经预定义的类型和类型联合(当你通过例子学习时,这些会更有意义)。这些并不是详尽无遗的,而只是可用类型的一个很好的样本。

\begin{longtable}[]{@{}
  >{\raggedright\arraybackslash}p{(\columnwidth - 2\tabcolsep) * \real{0.5000}}
  >{\raggedright\arraybackslash}p{(\columnwidth - 2\tabcolsep) * \real{0.5000}}@{}}
\toprule()
\begin{minipage}[b]{\linewidth}\raggedright
类型
\end{minipage} & \begin{minipage}[b]{\linewidth}\raggedright
描述
\end{minipage} \\
\midrule()
\endhead
\texttt{term} &
这被定义为\texttt{any}。\texttt{term}代表任何有效的Elixir项,这也包括带有\texttt{\_}作为参数的函数。 \\
\texttt{boolean} & 这被定义为两种布尔类型的联合 -
\texttt{false | true}。\texttt{char}:这被定义为有效字符的范围:\texttt{0..0x10ffff}。注意\texttt{..}是范围操作符。 \\
\texttt{number} & 这被定义为整数和浮点数的联合 -
\texttt{integer | float}。 \\
\texttt{binary} & 用这个表示Elixir字符串。 \\
\texttt{char\_list} &
用这个表示Erlang字符串。这被定义为\texttt{[char]}。 \\
\texttt{list} &
这被定义为\texttt{[any]}。你总是可以约束列表的类型。例如,\texttt{[number]}。 \\
\texttt{fun} &
\texttt{(... -> any)}表示\emph{任何}匿名函数。你可能想要根据函数的元数和返回类型来约束这个。例如,\texttt{(() -> integer)}是一个返回整数的元数为零的匿名函数,而\texttt{(integer, atom -> [boolean])}是一个元数为二的函数,它分别接受一个整数和一个原子,并返回一个布尔值列表。 \\
\texttt{pid} & 进程id \\
\texttt{tuple} &
任何类型的元组。其他有效的选项是\mintinline{elixir}|{}|和\mintinline{elixir}|{:ok, binary}|。 \\
\texttt{map} &
任何类型的映射。其他有效的选项是\texttt{\%\{\}}和\texttt{\%\{atom => binary\}}。 \\
\bottomrule()
\end{longtable}

表 10.1 一些可用于类型规范的类型

接下来的几个例子会让你有更好的感觉。

\begin{example}{加法}
\end{example}

让我们从一个简单的加法函数开始,这个函数接受两个数字并返回另一个数字。这是一种可能的类型规范:

\begin{code}{add/2的一种可能的类型规范}

\begin{minted}[linenos]{elixir}
@spec add(integer, integer) :: integer
def add(x, y) do
  x + y
end
\end{minted}
% \label{lst:id}
\end{code}

就目前而言,\texttt{add/2}可能过于严格。我们可能还想包括浮点数\emph{或}整数。写的方式如下:

\begin{code}{包括浮点数和整数作为输入参数和返回值}

\begin{minted}[linenos]{elixir}
@spec add(integer | float, integer | float) :: integer | float
def add(x, y) do
  x + y
end
\end{minted}
% \label{lst:id}
\end{code}

幸运的是,我们可以使用内置的简写类型\texttt{number},它被定义为\texttt{integer | float}。\texttt{|}表示\texttt{number}是一个联合类型。顾名思义,联合类型是由两种或多种类型组成的类型。联合类型可以应用于输入类型和返回值的类型。

\begin{code}{使用number简写表示integer \textbar{} float}

\begin{minted}[linenos]{elixir}
@spec add(number, number) :: number
def add(x, y) do
  x + y
end
\end{minted}
% \label{lst:id}
\end{code}

我们将在学习如何定义自己的类型时,很快看到更多的联合类型的例子。

\begin{example}{List.fold/3}
\end{example}

让我们尝试解决一些更具挑战性的问题:\texttt{List.fold/3}。这个函数通过一个函数从左边减少给定的列表。它还需要一个累加器的初始值。这是函数的工作方式:

\begin{code}{}
\begin{minted}[linenos]{elixir}
iex > List.foldl([1, 2, 3], 10, fn x, acc -> x + acc end)
\end{minted}
% \label{lst:id}
\end{code}

如预期,函数将返回\texttt{16}。第一个参数是列表,然后是累加器的初始值。最后一个参数是执行每一步减少的函数。这是函数参数(取自\texttt{List}源代码):

\begin{code}{List.foldl的函数参数}

\begin{minted}[linenos]{elixir}
def foldl(list, acc, function)
    when is_list(list) and is_function(function) do
  # the implementation is not important here
end
\end{minted}
% \label{lst:id}
\end{code}

\texttt{List.foldl/3}已经将\texttt{list}的类型限制为列表,这是由于\texttt{is\_list/1}守卫子句。然而,列表的元素可以是任何有效的Elixir项。同样,\texttt{function}需要是一个实际的函数。\texttt{function}必须是二元的,其中第一个参数的类型与\texttt{elem}相同,第二个参数的类型与\texttt{acc}相同。最后,这个函数的返回结果应该与\texttt{acc}的类型相同。指定\texttt{List.foldl/3}的类型规范的一种可能的方式可能是:

\begin{code}{编写List.foldl/3类型规范的一种可能(但不是很有帮助)的方式}

\begin{minted}[linenos]{elixir}
@spec foldl([any], any, (any, any -> any)) :: any
def foldl(list, acc, function)
    when is_list(list) and is_function(function) do
  # the implementation is not important here
end
\end{minted}
% \label{lst:id}
\end{code}

虽然从Dialyzer的角度来看,这个类型规范在技术上没有什么问题,但它并没有显示输入参数和返回值之间的类型关系。我们可以使用没有限制的类型变量作为函数的参数,如下所示:

\texttt{@spec function(arg) :: arg when arg: var}
注意\texttt{var},它表示任何变量。因此,我们可以向类型规范提供更好的变量名,如下所示:

\begin{code}{在类型规范中提供更好的变量名}

\begin{minted}[linenos]{elixir}
@spec foldl([elem], acc, (elem, acc -> acc)) :: acc
      when elem: var, acc: var
def foldl(list, acc, function)
    when is_list(list) and is_function(function) do
  # the implementation is not important here
end
\end{minted}
% \label{lst:id}
\end{code}

\begin{example}{映射函数}
\end{example}

我们也可以使用守卫来限制作为函数参数的类型变量,如下所示:

\texttt{@spec function(arg) :: arg when arg: atom}
在这个例子中,我们有自己的\texttt{Enum.map/2}实现。创建\texttt{lib/my\_enum.ex}。注意单个参数和返回结果的类型规范。

\begin{code}{映射函数的类型规范 (lib/my\_enum.ex0}

\begin{minted}[linenos]{elixir}
defmodule MyEnum do
  @spec map(f, list_1) :: list_2
        when f: (a -> b),
             list_1: [a],
             list_2: [b],
             a: term,
             b: term
  def map(f, [h | t]), do: [f.(h) | map(f, t)]

  def map(f, []) when is_function(f, 1), do: []
end
\end{minted}
% \label{lst:id}
\end{code}

从类型规范中,我们声明:

\begin{itemize}
\item  \texttt{f}(\texttt{map/2}的第一个参数)是一个单元函数,它接受一个项并返回另一个项。
\item  \texttt{list\_1}(\texttt{map/2}的第二个参数)和\texttt{list\_2}(\texttt{map/2}的返回结果)是项的列表。
\end{itemize}

我们也费了一番功夫来命名\texttt{f}的输入和输出类型。虽然这并不是严格必要的,但明确地放置\texttt{a}和\texttt{b}表示\texttt{f}在类型\texttt{a}上操作并返回类型\texttt{b},并且\texttt{map/2}接受类型\texttt{a}的列表作为输入并输出类型\texttt{b}的列表。如你所见,类型规范可以传达很多信息。

\section{编写你自己的类型}

你可以使用\texttt{@type}定义你自己的类型。例如,让我们为RGB颜色代码创建一个自定义类型。创建\texttt{lib/hexy.ex}:

\begin{code}{使用@type定义自定义类型 (lib/hexy.ex)}

\begin{minted}[linenos]{elixir}
defmodule Hexy do
  # 1
  @type rgb() :: {0..255, 0..255, 0..255}
  # 2
  @type hex() :: binary

  # 3
  @spec rgb_to_hex(rgb) :: hex
  def rgb_to_hex({r, g, b}) do
    [r, g, b]
    |> Enum.map(fn x -> Integer.to_string(x, 16) |> String.rjust(2, ?0) end)
    |> Enum.join()
  end
end

# 1 RGB颜色代码的类型别名
# 2 Hex颜色代码的类型别名
# 3 在规范中使用自定义类型定义
\end{minted}
% \label{lst:id}
\end{code}


我们本可以只指定\texttt{@spec rgb\_to\_hex(tuple) :: binary},但这并不能传达很多信息,也不能对输入参数进行很多约束,除了说预期一个元组。在这种情况下,甚至一个空的元组都是可以接受的。

相反,我们指定了一个有三个元素的元组,并进一步指定每个元素都是范围在0到255的整数。最后,我们给类型一个描述性的名字,比如\texttt{rgb}。对于\texttt{hex},我们没有简单地称之为\texttt{binary}(在Elixir中是一个字符串),而是将其别名为\texttt{hex},以便更具描述性。

\subsection{多重返回类型和无主体函数子句}

函数由多个返回类型组成是很常见的。在这种情况下,我们可以使用\emph{无主体函数子句}来将类型注解组合在一起。考虑以下情况:

\begin{code}{使用无主体函数子句并将类型规范附加到该子句上 (lib/hexy.ex)}

\begin{minted}[linenos]{elixir}
defmodule Hexy do
  @type rgb() :: {0..255, 0..255, 0..255}
  @type hex() :: binary

  @spec rgb_to_hex(rgb) :: hex | {:error, :invalid}
  # 1
  def rgb_to_hex(rgb)

  def rgb_to_hex({r, g, b}) do
    [r, g, b]
    |> Enum.map(fn x -> Integer.to_string(x, 16) |> String.rjust(2, ?0) end)
    |> Enum.join()
  end

  def rgb_to_hex(_) do
    {:error, :invalid}
  end
end

# 1 无主体函数子句
\end{minted}
% \label{lst:id}
\end{code}


这次,\texttt{rgb\_to\_hex/1}有两个子句。第二个子句是后备情况。这个后备情况总是会返回\mintinline{elixir}|{:error, :invalid}|。这意味着我们必须更新我们的类型规范。

我们可以创建一个\emph{无主体}函数子句,而不是像我们在前一个例子中那样在第一个函数子句上面写它。需要注意的一点是我们如何定义子句。这样会起作用:

\texttt{def rgb\_to\_hex(rgb)}
而这样\emph{不会}起作用:

\texttt{def rgb\_to\_hex(\{r, g, b\})}
如果你试图编译文件,你会得到一个错误消息:

\texttt{** (CompileError) lib/hexy.ex:7: can use only variables and \\\\ as arguments of bodiless clause}
有一个无主体函数子句可以将所有可能的类型规范集中在一个地方,这样就可以避免在每个函数子句上都撒上类型规范。


\subsection{揭示Elixir中的类型,第二部分}

除了\texttt{i/1},还有另一个方便的\texttt{iex}助手:\texttt{t/1}。\texttt{t/1}打印给定模块或给定函数/元数对的类型。如果你想了解模块中使用的类型(可能是自定义的)的更多信息,这很方便。例如,让我们研究一下在\texttt{Enum}中找到的类型:

\begin{code}{}
\begin{minted}[linenos]{elixir}
iex > t(Enum)
@type t() :: Enumerable.t()
@type element() :: any()
@type index() :: non_neg_integer()
@type default() :: any()
\end{minted}
% \label{lst:id}
\end{code}

在这里,我们可以看到\texttt{Enum}有四个定义的类型。\texttt{Enumerable.t}看起来很有趣。\texttt{Enumerable}模块也有一堆定义的类型:

\begin{code}{}
\begin{minted}[linenos]{elixir}
iex > t(Enumerable)
@type acc() :: {:cont, term()} | {:halt, term()} | {:suspend, term()}
@type reducer() :: (term(), term() -> acc())
@type result() :: {:done, term()} | {:halted, term()} | {:suspended, term(), continuation()}
@type continuation() :: (acc() -> result())
@type t() :: term()
\end{minted}
% \label{lst:id}
\end{code}

\subsection{回到Bug \#5}

在本章结束之前,让我们如约回到\texttt{Cashy.Bug5}。没有任何类型规范,Dialyzer无法找到明显的错误。然而,现在让我们添加类型规范:

\begin{code}{添加类型规范到Cashy.Bug5 (lib/bug\_5.ex)}

\begin{minted}[linenos]{elixir}
defmodule Cashy.Bug5 do
  @type currency() :: :sgd | :usd

  @spec convert(currency, currency, number) :: number
  def convert(:sgd, :usd, amount) do
    amount * 0.70
  end

  @spec amount({:value, number}) :: number
  def amount({:value, value}) do
    value
  end

  def run do
    convert(:sgd, :usd, amount({:value, :one_million_dollars}))
  end
end
\end{minted}
% \label{lst:id}
\end{code}

这次当我们运行Dialyzer时,它显示了一个我们\emph{没有}预期的错误,以及一个我们之前预期但没有得到的错误:

\begin{code}{}
\begin{minted}[linenos]{elixir}
bug_5.ex:22: The specification for 'Elixir.Cashy.Bug5':convert/3 states that the function might also return integer() but the inferred return is float()

bug_5.ex:32: Function run/0 has no local return
bug_5.ex:33: The call 'Elixir.Cashy.Bug5':amount({'value','one_million_dollars'}) breaks the contract ({'value',number()}) -> number()
done in 0m1.05s
done (warnings were emitted)
\end{minted}
% \label{lst:id}
\end{code}

让我们先处理第二个,更直接的错误。由于我们传入的是一个原子(\texttt{:one\_million\_dollars})而不是一个数字,Dialyzer正确地抱怨。

那么第二个错误呢?它说我们的类型规范表明函数可能会返回一个\texttt{integer},但Dialyzer推断出的是函数只返回\texttt{float}。现在当我们检查函数的主体时,我们看到:

\texttt{amount * 0.70}
当然!与浮点数相乘总是会返回一个浮点数!这就是为什么Dialyzer会抱怨。这很好,因为Dialyzer能够在某些情况下检查我们的类型规范是否存在明显的错误。

\section{练习}

\begin{enumerate}
\def\labelenumi{\arabic{enumi}.}
\item
  尝试使用\texttt{Cashy.Bug1}到\texttt{Cashy.Bug5},并尝试添加错误的类型规范。看看错误消息是否对你有意义。一个更难的练习是设计一个有明显错误的代码,但Dialyzer无法捕获这个错误。这是我们在\texttt{Cashy.Bug5}中做过的事情。
\item
  想象你正在编写一个纸牌游戏。一张牌由花色和值组成。为牌、花色和牌的值提出类型。让你开始:
\end{enumerate}

\begin{code}{}
\begin{minted}[linenos]{elixir}
@type card :: {suit(), value()} 
@type suit :: <FILL THIS IN> 
@type value :: <FILL THIS IN>
\end{minted}
% \label{lst:id}
\end{code}

\begin{enumerate}
\def\labelenumi{\arabic{enumi}.}
\setcounter{enumi}{2}

\item
  尝试为一些内置函数指定类型。一个好的开始是\texttt{List}和\texttt{Enum}模块。一个好的灵感来源是Erlang/OTP(是的,Erlang!)的代码库。语法稍有不同,但不应该对你构成主要的障碍。
\end{enumerate}

\section{总结}

Dialyzer已经在生产中得到了很好的效果。例如,它发现了OTP中以前未被发现的软件差异。虽然它不是银弹,但Dialyzer提供了一些静态类型检查器的优点,比如Haskell。

在你的函数中包含类型不仅可以作为文档,还可以让Dialyzer在发现差异时更准确。作为一个额外的好处,Dialyzer也可以指出你在类型规范中是否犯了错误。在本章中,我们学习了:

\begin{itemize}

\item
  成功类型,Dialyzer使用的类型推断机制
\item
  如何使用Dialyzer并解释它有时难以理解的错误消息
\item
  通过提供类型规范和守卫,如\texttt{is\_function(f, 1)}和\texttt{is\_list(l)},如何提高Dialyzer的准确性
\end{itemize}

在下一章中,我们将看一下为Erlang生态系统专门编写的测试工具。这些工具不是普通的单元测试工具。
这些强大的工具可以根据你定义的一般属性生成测试用例,并找出并发错误。
